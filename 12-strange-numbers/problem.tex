\documentclass[12pt]{report}
\usepackage[a4paper, total={7.3in, 9.7in}]{geometry}
\usepackage{amsmath}
\usepackage{upquote}
\usepackage{listings}
\usepackage{xcolor}
\usepackage{titlesec}
\usepackage{amssymb}

\definecolor{backgroundcolor}{rgb}{1, 1, 1}
\definecolor{commentstyle}{rgb}{0.365, 0.422, 0.475}
\definecolor{keywordstyle}{rgb}{0.6, 0.14, 0.576}
\definecolor{numberstyle}{rgb}{0.5, 0.5, 0.5}
\definecolor{stringstyle}{rgb}{0.77, 0.1, 0.08}

\lstdefinestyle{xcodecolor}{
    backgroundcolor=\color{backgroundcolor},   
    commentstyle=\color{commentstyle},
    keywordstyle=\color{keywordstyle},
    numberstyle=\scriptsize\color{numberstyle},
    stringstyle=\color{stringstyle},
    basicstyle=\ttfamily\footnotesize,
    breakatwhitespace=false,         
    breaklines=true,                 
    captionpos=b,                    
    keepspaces=true,                   
    numbersep=5pt,                  
    showspaces=false,                
    showstringspaces=false,
    showtabs=false,                  
    tabsize=2
}

\lstset{style=xcodecolor}

\usepackage[T1]{fontenc}
\usepackage{cascadia-code}

% Raised Rule Command:
%  Arg 1 (Optional) - How high to raise the rule
%  Arg 2            - Thickness of the rule
\newcommand{\raisedrule}[2][0em]{\leaders\hbox{\rule[#1]{1pt}{#2}}\hfill}

\setlength{\parindent}{0pt}
\titleformat{\section}
  {\normalfont\Large\bfseries}{\thesection}{1em}{}[{\titlerule[0.8pt]}]
\begin{document}

	{\Large
	\textbf{Strange Numbers}}
	
	\vspace{0.4cm}
	DiPS CodeJam 22\raisedrule[0.25em]{1pt}
	\\
	% document

  \section*{Prompt}
  Pranav and Prithvi are playing a game of \textit{Strange Numbers}. Pranav gives Prithvi 2 numbers, $x$ and $k$. Prithvi now needs to determine whether there is an integer $a$, such that it has $x$ positive divisors and exactly $k$ of them are prime numbers. Can you help him?

  \subsection*{Input Format}
  The first and only line of the input will contain 2 space-separated integers in the format $x, k$.
  \subsection*{Output Format}
  The first and only line of your output must contain: 1, if an integer $a$ exists, or 0 if it does not.
  \subsection*{Constraints}
  $1 \le a \le 1000$
  \subsection*{Sample Input/Output}
  \begin{tabular}{ |l|l| } 
    \hline
    \textbf{Input} & \textbf{Output} \\
    {\lstinputlisting{./testCases/input/input00.txt}} & {\lstinputlisting{./testCases/output/output00.txt}} \\ % use {\lstinputlisting{./testCases/input/input00.txt}} & {\lstinputlisting{./testCases/output/output00.txt}}
    \hline
   \end{tabular}


  \section*{Solution}
  Let's assume that a valid $a$ exists. The prime factorisation of $a$ would be $\prod_{i=1}^{k} p_i^{a_{i}}$ where $p_i$ are the prime factors of $a$ and $a_i$ are the exponents. Then, it is known that the number of factors of $a$ are $\prod_{i=1}^{k} (a_{i}+1)$. Hence, we have $x=\prod_{i=1}^{k} (a_{i}+1)$ for $a_i\ge1$, hence $(a_i+1)\ge2$.
  Thus, valid choice for $a$ exists if the prime factorization of $x$ has at least $k$ terms. So, we just need to compute prime factorization of $x$.


	\section*{Sample Program}
	\lstinputlisting[language=Python]{sampleSolution.py}
	

\end{document}