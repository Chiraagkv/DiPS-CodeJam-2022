\documentclass[12pt]{report}
\usepackage[a4paper, total={7.3in, 9.7in}]{geometry}
\usepackage{amsmath}
\usepackage{upquote}
\usepackage{listings}
\usepackage{xcolor}

\definecolor{backgroundcolor}{rgb}{1, 1, 1}
\definecolor{commentstyle}{rgb}{0.365, 0.422, 0.475}
\definecolor{keywordstyle}{rgb}{0.6, 0.14, 0.576}
\definecolor{numberstyle}{rgb}{0.5, 0.5, 0.5}
\definecolor{stringstyle}{rgb}{0.77, 0.1, 0.08}

\lstdefinestyle{xcodecolor}{
    backgroundcolor=\color{backgroundcolor},   
    commentstyle=\color{commentstyle},
    keywordstyle=\color{keywordstyle},
    numberstyle=\scriptsize\color{numberstyle},
    stringstyle=\color{stringstyle},
    basicstyle=\ttfamily\footnotesize,
    breakatwhitespace=false,         
    breaklines=true,                 
    captionpos=b,                    
    keepspaces=true,                   
    numbersep=5pt,                  
    showspaces=false,                
    showstringspaces=false,
    showtabs=false,                  
    tabsize=2
}

\lstset{style=xcodecolor}

\usepackage[T1]{fontenc}
\usepackage{cascadia-code}

% Raised Rule Command:
%  Arg 1 (Optional) - How high to raise the rule
%  Arg 2            - Thickness of the rule
\newcommand{\raisedrule}[2][0em]{\leaders\hbox{\rule[#1]{1pt}{#2}}\hfill}
\setlength{\parindent}{0pt}
\usepackage{titlesec}
\usepackage{amssymb}
\titleformat{\section}
  {\normalfont\Large\bfseries}{\thesection}{1em}{}[{\titlerule[0.8pt]}]
\begin{document}

	{\Large
	\textbf{Cricket Mania}}
	
	\vspace{0.4cm}
	DiPS CodeJam 22\raisedrule[0.25em]{1pt}
	\\


	\section*{Prompt}
    A cricket championship of \textit{n} teams is taking place, where each team plays all other teams once in a \textit{round-robin} fashion. A team gets 5 points for winning a match, and 0 for loosing. It is assumed that no match will end in a draw or a tie. What is the \textbf{maximum possible difference} of points between the winning team and the second-placed team?
    
    \subsection*{Input Format}
    \begin{itemize}
      \item The first line of input contains a single integer $t$, denoting the number of test cases.
      \item The next $t$ lines of input contain a single integer $n$.
    \end{itemize}
    \subsection*{Output Format}
    For each test case, output in a single line the maximum difference of points between first and second place.
    \subsection*{Constraints}
    \begin{itemize}
      \item $ 2 \le n \le 50 $
      \item $ 1 \le t \le 50 $
    \end{itemize}
    \subsection*{Sample Input/Output}
    \begin{tabular}{ |l|l| } 
      \hline
      \textbf{Input} & \textbf{Output} \\
      {\lstinputlisting{./testCases/input/input10.txt}} & {\lstinputlisting{./testCases/output/output10.txt}} \\ % use \lstinputlisting{./testCases/<in/out>put/00/.txt}
      \hline
     \end{tabular}

    \section*{Solution}
    \subsection*{Simplifying the Problem}
    We can simplify the problem like so:\\
    Find the maximum number of points by which a team can win a \textit{n}-team competition, if each team plays every other team in round robin fashion. Teams get 5 points for a win and 0 points for a loss. Assume that there are no draws.
    \subsection*{Solving the Problem}
    To get the maximum possible points, a team has to win every match they play. Let's assume that the team in 1st place won all the matches they played. In a \textit{round-robin} fashion, each team will have played $n-1$ matches. As the first team won all the matches they played, all the other teams have $n-2$ matches, which they may win or loose. To maximize the difference in points between the first-placed team and second-placed team, points of the second-placed team must be the minimum possible, without changing its standing. This will happen only when the team wins and looses an equal number of matches. Applying this, we get the formula:
    \begin{align*}
    r (\text{number of rounds to play}) &= n-2 \\
    \text{minimum possible wins} &= \lceil\frac{r}{2}\rceil = \lceil\frac{n-2}{2}\rceil
    \end{align*}
    Therefore ($\text{points/win}=5$):
    \begin{align*}
      \text{points won by the first-placed team} &= 5\cdot(n-1) \\
      \text{points won by the second-placed team} &= 5\cdot(\lceil\frac{n-2}{2}\rceil)\\
      \therefore\text{difference in points} &= 5\cdot(n-1) - 5\cdot(\lceil\frac{n-2}{2}\rceil)\\
      &= \texttt{(5 * (n - 1)) - (5 * ceil((n - 2) / 2))}
    \end{align*}
    We will use this formula in the sample solution given below.

	\section*{Sample Program}
	\lstinputlisting[language=Python]{sampleSolution.py}
	

\end{document}