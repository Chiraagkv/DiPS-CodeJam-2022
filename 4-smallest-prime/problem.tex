\documentclass[12pt]{report}
\usepackage[a4paper, total={7.3in, 9.7in}]{geometry}
\usepackage{amsmath}
\usepackage{upquote}
\usepackage{listings}
\usepackage{xcolor}
\usepackage{titlesec}
\usepackage{amssymb}

\definecolor{backgroundcolor}{rgb}{1, 1, 1}
\definecolor{commentstyle}{rgb}{0.365, 0.422, 0.475}
\definecolor{keywordstyle}{rgb}{0.6, 0.14, 0.576}
\definecolor{numberstyle}{rgb}{0.5, 0.5, 0.5}
\definecolor{stringstyle}{rgb}{0.77, 0.1, 0.08}

\lstdefinestyle{xcodecolor}{
    backgroundcolor=\color{backgroundcolor},   
    commentstyle=\color{commentstyle},
    keywordstyle=\color{keywordstyle},
    numberstyle=\scriptsize\color{numberstyle},
    stringstyle=\color{stringstyle},
    basicstyle=\ttfamily\footnotesize,
    breakatwhitespace=false,         
    breaklines=true,                 
    captionpos=b,                    
    keepspaces=true,                   
    numbersep=5pt,                  
    showspaces=false,                
    showstringspaces=false,
    showtabs=false,                  
    tabsize=2
}

\lstset{style=xcodecolor}

\usepackage[T1]{fontenc}
\usepackage{cascadia-code}

% Raised Rule Command:
%  Arg 1 (Optional) - How high to raise the rule
%  Arg 2            - Thickness of the rule
\newcommand{\raisedrule}[2][0em]{\leaders\hbox{\rule[#1]{1pt}{#2}}\hfill}

\setlength{\parindent}{0pt}
\titleformat{\section}
  {\normalfont\Large\bfseries}{\thesection}{1em}{}[{\titlerule[0.8pt]}]
\begin{document}

	{\Large
	\textbf{Smallest Prime}}
	
	\vspace{0.4cm}
	DiPS CodeJam 22\raisedrule[0.25em]{1pt}
	\\
	% document

  \section*{Prompt}
  Given a $3\times3$ 2D array of integers as input, return the smallest prime number in the array. Return \texttt{none} if there is no prime number.

  \subsection*{Input Format}
  The input contains a $3\times3$ 2D array of integers.
  \subsection*{Output Format}
  Your output should be a single integer. If there are no prime numbers, print \texttt{none}.
  \subsection*{Constraints}
  \begin{itemize}
    \item $ 1 \le n \le 100 $
  \end{itemize}
  \subsection*{Sample Input/Output}
  \begin{tabular}{ |l|l| } 
    \hline
    \textbf{Input} & \textbf{Output} \\
    {\lstinputlisting{./testCases/input/input00.txt}} & {\lstinputlisting{./testCases/output/output00.txt}} \\ % use \lstinputlisting{./testCases/<in/out>put/<in/out>put00.txt}
    \hline
   \end{tabular}


  \section*{Solution}
  Let's take a look at the sample input:\\
  \lstinputlisting{./testCases/input/input00.txt}
  First, let's filter out the primes:\\
  \texttt{51, 19}\\
  At this point, if we have no items in the array, we can print \texttt{none} and exit. Let's sort the array:\\
  \texttt{19, 51}\\
  The first element of the array is \texttt{19}. Thus, we can conclude theat the smallest prime number in the input is $19$.\\
\\

	\section*{Sample Program}
	\lstinputlisting[language=Python]{sampleSolution.py}
	

\end{document}