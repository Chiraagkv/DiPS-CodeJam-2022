\documentclass[12pt]{report}
\usepackage[a4paper, total={7.3in, 9.7in}]{geometry}
\usepackage{amsmath}
\usepackage{upquote}
\usepackage{listings}
\usepackage{xcolor}
\usepackage{titlesec}
\usepackage{amssymb}

\definecolor{backgroundcolor}{rgb}{1, 1, 1}
\definecolor{commentstyle}{rgb}{0.365, 0.422, 0.475}
\definecolor{keywordstyle}{rgb}{0.6, 0.14, 0.576}
\definecolor{numberstyle}{rgb}{0.5, 0.5, 0.5}
\definecolor{stringstyle}{rgb}{0.77, 0.1, 0.08}

\lstdefinestyle{xcodecolor}{
    backgroundcolor=\color{backgroundcolor},   
    commentstyle=\color{commentstyle},
    keywordstyle=\color{keywordstyle},
    numberstyle=\scriptsize\color{numberstyle},
    stringstyle=\color{stringstyle},
    basicstyle=\ttfamily\footnotesize,
    breakatwhitespace=false,         
    breaklines=true,                 
    captionpos=b,                    
    keepspaces=true,                   
    numbersep=5pt,                  
    showspaces=false,                
    showstringspaces=false,
    showtabs=false,                  
    tabsize=2
}

\lstset{style=xcodecolor}

\usepackage[T1]{fontenc}
\usepackage{cascadia-code}

% Raised Rule Command:
%  Arg 1 (Optional) - How high to raise the rule
%  Arg 2            - Thickness of the rule
\newcommand{\raisedrule}[2][0em]{\leaders\hbox{\rule[#1]{1pt}{#2}}\hfill}

\setlength{\parindent}{0pt}
\titleformat{\section}
  {\normalfont\Large\bfseries}{\thesection}{1em}{}[{\titlerule[0.8pt]}]
\begin{document}

	{\Large
	\textbf{Classes, Classes!}}
	
	\vspace{0.4cm}
	DiPS CodeJam 22\raisedrule[0.25em]{1pt}
	\\
	% document

  \section*{Prompt}
  It's a long day, and Guru has a lot of activities to attend. He needs to select the maximum number of activities that he can do in a given time frame, assuming that he can only work on a single activity at a time. Each activity has a set start and end time.\\
  Can you help him figure out how many activities he can attend?

  \subsection*{Input Format}
  \begin{itemize}
    \item The first line of the input contains an integer $n$, denoting the number of activities.
    \item The next $n$ lines of the input each contain the start and end times of an activity, in the format $(\text{start}, \text{end})$.
  \end{itemize}
  \subsection*{Output Format}
  The first and only line of your output must contain a single integer $m$, denoting the maximum number of activities he can attend.
  \subsection*{Constraints}
  \begin{itemize}
    \item $ 4 \le n \le 24 $
    \item Assume that the activities are already sorted based on end times.
  \end{itemize}
  \subsection*{Sample Input/Output}
  \begin{tabular}{ |l|l| } 
    \hline
    \textbf{Input} & \textbf{Output} \\
    {\lstinputlisting{./testCases/input/input00.txt}} & {\lstinputlisting{./testCases/output/output00.txt}} \\ % use {\lstinputlisting{./testCases/input/input00.txt}} & {\lstinputlisting{./testCases/output/output00.txt}}
    \hline
   \end{tabular}


  \section*{Solution}
   This is an example of the \textit{Activity Selection Problem}.
  \subsection*{Simplifying the Problem}
  Assume there exist $n$ activities with each of them being represented by a start time $s_i$ and finish time $f_i$. Two activities $i$ and $j$ are said to be non-conflicting if $s_i \ge f_j$ or $s_j \ge f_i$. The activity selection problem consists in finding the maximal solution set (S) of non-conflicting activities. Here, using a greedy algorithm to find the solution will always result in an optimal solution.
  \subsection*{Solving the Problem}
   \begin{itemize}
    \item Let us create an empty array \texttt{arr}.
    \item Now we can start adding activities to this array.
    \item Since this is a greedy algorithm, the first activity is always selected.
    \item Now we loop through the rest of the activities. For each activity:
    \begin{itemize}
      \item If this activity has a start time that is greater than or equal to the finish time of the previously selected activity, then append it to \texttt{{arr}}.
     \end{itemize}
     \item Finally, we print the length of \texttt{arr}, denoting the number of activities.
   \end{itemize}

	\section*{Sample Program}
	\lstinputlisting[language=Python]{sampleSolution.py}
	

\end{document}